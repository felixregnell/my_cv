%!TEX encoding = UTF-8 Unicode

\documentclass{cv}

\title{\vspace{-0.5em}\textit{Curriculum Vitae}\\\textbf{Felix Regnell}}
\date{\vspace{-3.0em} 2024-11-22}

\begin{document}
\maketitle

%-----------CONTACT DETAILS------------------
% Make sure all the details are correct, you can add more links in the first row of second column if needed

\vspace{-2.1em}\begin{center}
	\textbf{Short Description}
\end{center}

\vspace{1.1em}\section{Contact me}

\begin{tabular*}{\textwidth}{p{5cm} l}
	Name           & Felix Kai Regnell \\
	Date of birth  & 2003-09-10 \\
	Address        & Dag Hammarskjölds Väg 3F, 224 64 Lund \\
	Email          & \url{felix.regnell@gmail.com} \\
	Mobile         & +46729757242
\end{tabular*}

\section{Ongoing Education}

\begin{tabular*}{\textwidth}{p{5cm} l}
	Education                  & MSc Eng, Computer Science and Engineering 	 \\
	University                 & Lund University, Lunds Tekniska Högskola (LTH) \\
	Current year               & 3rd	     \\
	Admission date             & 2022-08-22 \\
	Estimated graduation date  & 2027-06-06 \\
	Education Program& \url{https://www.lth.se/utbildning/datateknik300/}\\ \\
\end{tabular*}

I am currently taking a course in Efficient C programming (course code EDAG01), extra credits, in order to learn more about how high level programming languages are compiled to the computer architecture. I find the connection between hardware and software intriguing; how the code I write is optimized and executed; how to write high performance code according to hardware constraints and capabilities; how data structures are represented in memory and their complexity. I am thinking of specializing in embedded systems, and next year I am considering the following courses: Multicore Programming, Computer Architecture, and Compilers.

\section{Completed Education}
\begin{tabular*}{\textwidth}{p{5cm} l}
	Education        & Secondary school, Natural Science Program (3 years) \\
	School           & Polhemsskolan, Lund \\
	Graduation       & Spring 2022 \\
	More information & \url{https://lund.se/gymnasiewebbar/polhemskolan}\\ \\
\end{tabular*}

\vspace{1em} Transcripts of my university studies, and secondary school grades can be given out per request.

\vspace{2em}\hfill\textit{Continues on next page.}
\newpage
\section{Prior Experience}

\begin{tabular*}{\textwidth}{p{3.4cm}   p{3.5cm}  p{11cm}  }
	\textit{date}  & \textit{Company}  & \textit{Description} \\ \hline \\

	2018--2020 & Värpinge IF & \textbf{Parkour Coach/Trainer}. I coached children in the ages of 10-15 in the arts of Parkour, on a weekly basis and at 1-3 week long training camps. The job included planning of activities and responsibility over the children.\\

	2023--now & Lemongrass & \textbf{Service}. I work extra as a waiter about once a week at Lemongrass, a Restaurant in central Malmö. As a waiter I am responsible for customer satisfaction and coordination with other waiters, aswell as the kitchen staff. Reference per request.

\end{tabular*}

\section{Programming Languages}
\begin{tabular*}{\textwidth}{p{2.7cm}   p{3.7cm} p{11cm}  }
	\textit{Language} & \textit{Level} & \textit{Comment} \\ \hline \\
	
	Java  	& Advanced 	   & Multiple courses (27hp in total)  \\
	Scala 	& Intermediate & Introduced to programming through scala (7.5hp)\\
	Haskell & Basic        & Functional programming course (5 hp) \\
	C 		& Basic        & Currently taking a course in efficient C programming \\
	C++ 	& Basic        & As extra credit, laboratory work and assignment (so far 3.0 of 7.5hp) \\
	VHDL 	& Very basic   & For a course in digital design, using Verilog to simulate hardware for a simple CPU on an FPGA (9 hp) \\
	
	\end{tabular*}

\section{Languages}

\begin{tabular*}{\textwidth}{p{2.7cm}   p{3.7cm}  }
	\textit{Language} & \textit{Level}  \\ \hline
	Swedish & Native  			\\
	English & Fluent  			\\
	German  & Conversational	\\
\end{tabular*}

\end{document}